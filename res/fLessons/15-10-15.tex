\section{Lezione del 15-10-15}

\subsection{Differenze tra designer e utenti}

In gergo tecnico quando si inseriscono in un sito web troppi elementi che rovinano la navigazione si ha il \textit{bloated design} e per gli utenti pu\`o risultare una terribile esperienza. Statisticamente \`e stato dimostrato che \`e estremamente fastidioso e, oltre che per ragioni estetiche, anche per ragioni pratiche (ovvero aumentano le difficolt\`a computazionali).

Con la guerra dei browser sono stati introdotti molti effetti non standard nell'HTML dai vari produttori di software per rendere il proprio browser pi\`u competitivo e ricco di feature. L'escalation \`e cominciata con Lou Montulli, che co-invent\`o il \textit{blink tag}\footnote{Di cui poi se ne pent\`i}.

Nella grande famiglia del ``bloated design'', ci sono anche gli \textit{abusi da multimedia}, come per esempio l'utilizzo dell'audio o delle interfacce tridimensionali, che causano molta scena ma l'usabilit\`a e la soddisfazione finale ne risentono in quanto il costo computazionale che ne risulta per l'utente \`e troppo alto.

\paragraph*{BumpTop} L'idea di prendere spunto dalla nostra realt\`a e di ripercuoterla sui nostri sistemi informativi non paga perch\`e nonostante all'apparenza ci sia un costo computazionale ci\`o non si verifica: infatti si ha ad una bassa usabilit\`a e risulta difficile iteragirci. Quindi, anche senza i limiti imposti dal Web, non \`e per nulla facile superare i problemi computazionali (anche derivati dall'inerzia). Il modo corretto per porsi su questi termini, invece delle 3d views (che sono un grande problema di usabilit\`a) conviene offrire \textit{snapshot 2d} di oggetti 3d. Lo snapshot \`e sempre 2d (bassa complessit\`a).

\paragraph*{Plug-in} Un altro multimedia usabile \`e il plug-in, che soffre di un problema fondamentale: essendo ``plug-in'' appunto non \`e standard, e agli utenti medi non piace installare plugin per un discorso di \textit{trust}: \`e stato dimostrato che si dovrebbe incentivare a usare tecnologie plug-in solo dopo un anno da quando l'utente visitante ha cominciato a frequentare il sito. Un secondo motivo per evitare l'uso di plug-in \`e la perdita di tempo che si causa all'utenza: si \`e notato che il 90\% degli utenti lascia il sito senza installare il plug-in quando costretti.

\paragraph*{Flash} Un'altra tecnologia basata sempre su plug-in \`e flash, che soffre degli stessi problemi di prima, con l'aggiunta dei problemi di caricamento\footnote{I blob flash possono diventare molto pesanti}. In tutto ci\`o sono da considerare i \textit{problemi estrinseci}: per i developer si ha un rischio altissimo di bloated design, in quanto son presenti molti tool per sviluppo.
