\section{Lezione del 10-12-15}

\subsubsection{Lato tecnico}

\paragraph*{URI}Gli URI si distinguono dagli URL, e non vanno confusi. Gli URI sono un sovrainsieme degli indirizzi web. Ad esempio un www.sito.it non \`e un URI corretto, e non esiste negli standard: la voce corretta sarebbe http://www.sito.it. URI sono anche indirizzi mail.

\paragraph*{Differenze tra URIs, URLs, URNs}
\begin{itemize}
  \item[URL] Gli URL stanno per \textit{Uniform Resource Locator}, ovvero gli URL attraverso la loro rappresentazione ci indicano il modo per arrivare alla risorsa. Quindi gli indirizzi ``http://...'' sono URLs.

\item[URN] URNs significante \textit{Uniform Resource Name} sono identificatori che restano unici e persistenti che durano anche quando la risorsa cessa di esistere o non \`e pi\`u disponibile. Hanno la forma tipica:
\begin{verbatim}
urn:NID:...
\end{verbatim}
Per esempio, gli identificativi ISBN sono URN, rappresentano un nome di un libro, che non \`e detto che sia ancora disponibile.

\item[URI] Un URI pu\`o essere \textit{assoluto} o \textit{relativo}. Se assoluto \`e gi\`a ``completo'' cos\`i com\`e, mentre se relativo non \`e completo e per essere completato deve essere trasformato in un URI assoluto tramite informazione derivante dal contesto. La loro struttura si rappresenta nella seguente maniera:
\begin{verbatim}
Struttura:parte-dipendente-dallo-schema
\end{verbatim}

Lo schema \`e quello che definisce la \textit{semantica} (il significato) dell'URI. Gli URI in generale possono essere parte di due grosse famiglie:
\begin{itemize}

\item[Gerarchici] Quando la loro forma \`e:
  \begin{verbatim}
Schema :// authority path ? query
\end{verbatim}

Le autortiry, rappresentano l'\textit{autorit\`a}, ovvero \`e l'elemento che indica l'autorit\`a, e che risponde in caso di problemi. La \textit{path}, ``cammino'', si compone di zero o pi\`u segmenti, ognuno della forma:
\begin{verbatim}
/segmento
\end{verbatim}
Nota: lo hash(\#) \`e un carattere riservato per gli URI, e serve per delimitare l'URI di un oggetto con un identificatore di un frammento interno alla risorsa considerata.
La \textit{query} presenta l'informazione interpretata dalla risorsa.

\item[Opachi]Gli URI possono essere anche opachi, quando manca il simbolo ``/''. Per esempio:
\begin{verbatim}
mailto:director@cnn.com
\end{verbatim}
rappresenta un URI opaco. Nota che anche in questo caso \`e possibile passare argomenti nella query. Altri esempi di URI possono essere i numeri di telefono (tel) o fax (fax) e altri.

  
\end{itemize}

\end{itemize}

Gli URI, URL, URN usano gli ASCII, ma dato l'espansione del suo utilizzo si \`e passati dall'URI all'IRI\footnote{Che sta per Internationalized Resource Indentifiers}, che adotta un set di catatteri pi\`u ampio. Da qui sono nati gli IDN, che possono essere usati anche nelle estensioni web. Tutto ci\`o ha portato ad un maggior rischio agli \textit{attacchi omografici}. Infatti \`e possibile che siano presenti indirizzi graficamente uguali ma i simboli provengano da altri alfabeti.

\paragraph*{Problema di Opacit\`a degli URI}Il significato di:
\begin{verbatim}
http://www.sito.it/a/b.html
\end{verbatim}
non indica necessariamente che il sito sia in italiano, o che il sito si trovi in italia, in quanto questi indicano solamente dei nomi. Queste informazioni infatti non sono fornite dall'URL. L'URL rappresenta una \textit{stringa opaca} da cui non \`e lecito inferire alcuna propriet\`a della risorsa corrispondente. HTTP fornisce metodi (\textit{content negotiation}) per capire il formato dati. La parte finale dell'URL non \`e attendibile.

Un altro problema riguada i TLD (\textit{Top Level Domain}). La tendenza \`e di sovradimensionare i TLD, creandone troppi. La proposta \`e iniziata col creare nuovi TLD per categorizzare siti per adulti o porno, tutto ci\`o spinto da aziende che vendono domini. Questo causa il forzare tutto il contenuto all'interno di un certo TLD (nel caso pensate), o di consigliarlo (nel caso leggero). Anche se i costi tecnologici sono bassi, il \textit{costo sociale} nel caso leggero non presenta alcun vantaggio e nel caso pesante l'interpretazione del WHO diventa pi\`u informativa, ma non eliminando i vari problemi di categorizzazione, perch\`e usare l'URL per scopi che non sono i suoi, forzandoci informazione che non dovrebbe stare l\`i, \`e pura stupidit\`a tecnica.


\subsection{Torre di babele}

Al MIT vennero creati dei sistemi intelligenti come ``Manda una rosa alla mia ragazza'', dove si veniva trovato un sito web di vendita fiori e venivano inviate delle rose all'indirizzo salvato. Questi sistemi non funzionano bene in quanto ci sono problemi con la raccolta di informazioni nel web.
Un sistema simile viene eseguito con la RIIA (tipo SIAE italiana) per la ricerca di musica piratata. Questi bot per\`o potrebbero ricadere nello stesso caso del MIT.
