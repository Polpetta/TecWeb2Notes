\section{Lezione del 08-01-16}

\begin{enumerate}

\item[]In generale, se proprio si vogliono usare icone (senza testo associato), occorre rispettare il principio dell'\textbf{explainability}: tenendo premuta l'icona si ottiene una sua descrizione. Inoltre va tenuto conto della \textbf{escapability}: posso sempre ``scappare'' (escape) da una azione semplicemente spostando il dito fuori dall'area (principio valido non solo per le icone ma in generale per qualunque azione touch).
Esistono delle taglie pubblicitarie definite dal mercato, ovvero dall'IAB\footnote{Interactive Advertising Bureau)}, che impongono delle taglie di 300x250 (IAB ``medium''), 486x60 (IAB ``full-size''), 728x90(IAB ``Leaderboard''). L'\textbf{interstitial ads} \`e una pubblicit\`a che si prende tutto lo schermo nel cellulare, e questa tecnologia \`e basata su HTML5\footnote{La loro usabilit\`a \`e in ogni caso pessima.}. Esistono poi i SmartBanners, che sono dei banner in cui la loro grandezza varia in base all'ampiezza dello schermo, e la loro usabilit\`a non viene vista male dagli utenti, che ``tollerano''. I SmartBanners funzionano molto male se sono implmenentati in posizione non fissa, e non vanno applicati effetti speciali. La SmartAppBanner ha un gradimento bassissimo, avendo un effetto pop-up.

\item[Interazione] I problemi delle dita \`e che sono un mezzo rudimentale e risultano ``grasse''. Il dito medio \`e largo 11 mm, e pu\`o arrivare fino a 19 mm. Quindi tutte le aree cliccabili devono essere grandi a sufficienza perch\`e si riescano a centrare con un dito. Viene stabilito quindi che la \textbf{taglia minima} di un'area cliccabile \`e di 7x7 millimetri, con una zona di ``padding'' di almeno 2 mm. Se si vuole soddisfare l'utente si dovrebbe fare l'area cliccabile di almeno 9x9. Pi\`u dell'83\% dei top 1000 siti al mondo non segue queste semplici linee guida, ed ha seri problemi di usabilit\`a. Per ovviare parzialmente a questi problmi occorre in ogni caso seguire sempre un importante principio: \textbf{Reversibility} $\to$ Ogni azione suscettibile di errore di puntamento dovrebbe essere \textit{reversibile}.
  Fitts vale ancora, ma bisogna considerare le prese del cellulare che sono 5: simetrica, assimetrica, singola e hanno diverse aree di usabilit\`a. Usare i pollici peggiora la precisione. La soluzione migliora tenendo i controlli nella parte bassa, che fanno aumentare l'usabilit\`a. Nota che \`e un problema ancora pi\`u importante con i \textbf{tablet}, dove tipicamente la navigazione \`e a due mani (minimizza lo sforzo muscolare). Se l'area non \`e portata di un pollice, le distanze diventano praticamente immense. Sia su smartphone che su tablet, la forma dello schermo (e delle aree raggiungibili) cambia a seconda che si usa la normale posizione oppure la modalit\`a \textbf{landscape}. Quindi la miglior interfaccia mobile tiene conto di tutti i casi, offrendo la possiblilt\`a di scegliere la modalit\`a d'uso. I punti ``magici'' in un telefono sono i \textbf{fun menu}, contrariamente alle pie menu che in questo caso non sono molto supportate. Anche le tastiere usano i fun menu, permettendo all'utente di vedere la lettera premuta. Questo si sta effettuando anche nelle tastiere swipe.

\end{enumerate}

\subsection{Social Network}

Due date importanti: 2010 e 2013.

Nel 2010 Facebook sorpassa Google, segnalando quindi che il Web Sociale supera il Web dell'Informazione. Una mensione, raccomandazione, segnalazione da parte di un amico o anche semplicemente da un'altra \textbf{persona} comune \`e pi\`u efficace di ordini di grandezza (10x - 100x!). Questa componenete cos\`i importante del social ha fatto si che nascessero dei problemi come:
\begin{itemize}

\item Belkin sorpresa a pagare utenti per postare fake reviews sui loro prodotti dando il massimo punteggio
\item 2013 Sumsung paga per fare inserire commenti e reviews social che promuovano i suoi smartphone e che denigrino HTC
\item Microsoft scoperta a pagare bloggers e utenti twitter per promuovere ``spontaneamente'' internet Explorer

\end{itemize}


Nel 2013 il traffico \textbf{non umano} (bots e algoritmi) ha \textbf{superato} quello \textbf{umano}. Attualmente il 10\% di tutta l'attivit\`a sui social \`e finta, frutto di automazione. A fine del 2016 si presume il 20\%. In twitter, solo all'incirca il 35\% \`e reale. Di tutti i reviews su Internet, \textbf{un terzo} sono falsi.

\subsubsection{Visibilit\`a}
Sfrutta la trasmissione dell'informazione (viralit\`a), che potenzialmente da uno speedup di diffusione \textbf{esponenziale}.
Esistono 4 regole virali di base, combinabili:
\begin{enumerate}

\item[4] \textbf{Social Currency}: contenuto che agisce da ``moneta'': rende chi lo condivide speciale, gli da una luce positiva (esempio, notizia ``segreta'', concetto di ``elite'' o di ``prestigio'').
\item[3] \textbf{Pratical Value}: valore pratico associato al contenuto (ad esempio sconti, recensioni, suggerimenti pratici).
\item[2] \textbf{Storytelling}: dare un contesto con una storia che attiri ed invogli a seguirne gli sviluppi.
\item[1] \textbf{Emotion}: soprattutto \`e importante dare emozioni! Non tutte le emozioni vanno bene, ed esiste un principio di base: le emozioni positive sono pi\`u virali delle emozioni negative. Analogamente, \`e il motivo, ad esempio, per cui Facebook non ha un ``dislike'' ma solo ``like''. Top ten emozionale
  \begin{enumerate}

  \item[10] Eccitazione
  \item[9] Affetto
  \item[8] Speranza
  \item[7] Piacere
  \item[6] Gioia
  \item[5] Delizia
  \item[4] Felicti\`a
  \item[3] Sorpresa
  \item[2] Interesse
  \item[1] Divertimento
  \end{enumerate}

  Top worst emozionali:
  \begin{enumerate}
    
  \item Rabbia
  \item Colpevolezza
  \item Disprezzo
  \item Vergogna
  \item Imbarazzo
  \item Cortesia
  \item Frustazione
  \item Disperazione
  \item Offensivit\`a
    
  \end{enumerate}

  La \textbf{Sentiment Analysis} studia continuamente la sensibilit\`a delle persone, prestando attenzione alle campagne sociali per monitorare costantemente i feedback. Tutta questa complessit\`a social sta man mano diventando sempre pi\`u simile ad un altro sistema complesso: la rete web e il cervello umano.
  
\end{enumerate}
