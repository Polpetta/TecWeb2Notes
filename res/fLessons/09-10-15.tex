\section{Lezione del 09-10-15}

Violare le condizioni web significa violare la prassi. Una condizione web infatti non \`e lo standard web: significa che \`e una tecnica usata dalla maggioranza dei siti web. Rispettare le convenzioni si collega ad una delle pi\`u conosciute leggi della usabilit\`a.

\textbf{Legge di Jakob}: Gli utenti spendono la maggior parte del loro tempo su \textit{altri} siti web. Quindi non \`e possibile (o meglio, non si dovrebbe) piegare gli utenti ai voleri dei designer: ci\`o infatti causa perdita di tempo e frustazione durante l'utilizzo.

Un altro dei gravi problemi di usibilit\`a \`e sempre legato all'asse What: ovvero usare \textit{un linguaggio vuoto o con poco contenuto - slogan}. L'utente che arriva ad una certa pagina si aspetta contenuto, non un linguaggio pomposo e difficile, che causa problemi di navigabilit\`a.

\textit{Usare testo difficile e monolitico} crea problemi nella navigazione. In generale, il testo web \`e diverso dal testo normale: oltre che per i timers, la lettura sul media schermo \`e pi\`u difficile, e necessita di semplificazioni. Regole di base:
\begin{itemize}

\item Dal 100\% del testo normale bisogna ridurlo del 50\%. Se l'audience \`e generalista bisognerebbe portarlo al 25\% (ovvero ridurlo a un quarto).

\item Conviene \textit{cominciare con la conclusione}\footnote{Ovvero partire direttamente dal punto focale del discorso}, e poi espandere il concetto.

\end{itemize}

I siti che solitamente hanno il monopolio assoluto su una determinata categoria (ad esempio i siti governativi) tendono a non rispettare queste regole.


\subsubsection{Non persistenti}

I problemi non persistensi sono quei problemi che hanno subito dei cambiamenti in male o in meglio nel tempo.

Le \textit{splash page}, usate soprattutto dai designer, vengono considerate negativamente dagli utenti, perch\`e gli fanno perdere tempo prezioso per raggiungere il loro obiettivo. Le splash page sono da evitare a tutti i costi: se sono anche animate causeranno una maggiore frustazione a chi naviga nel sito.

Un altro problema \`e la \textit{la richiesta delle informazioni}, che comporta uno sforzo agli utenti. La \textit{registrazione prematura} produce un'altissima perdita di utenti\footnote{Ci\`o non si applica se si ha una forte motivazione}, che non possono conoscere i benefici della registrazione prima di poter effettuare una visita al sito, inoltre la richiesta di password causa uno sforzo computazionale per inserire e poi ricordare la coppia username-password. Con ci\`o nasce anche il problema del \textit{trust}\footnote{Fiducia}: dare informazioni personali richiede un sito di cui mi fido, e gli utenti generalmente non si fidano. I dati dicono che la registrazione prematura porta ad una diminuzione di un ordine di grandezza degli utenti potenziali (meno di 1 su 10).

Lo \textit{scrolling} dipende da quanto gli utenti scrollano in media: 1,3 schermi in pi\`u rispetto a quello di cui stanno gi\`a vedendo (in totale quindi 2,3 schermi visualizzati), dopodich\`e tutto quello che c'\`e dopo non viene visualizzato pi\`u, in quanto causa frustazione. Lo scroll va usato con parsimonia: le probabilit\`a di scroll per un utente che visita per la prima visita un'homepage \`e pari al 23\%. Per le pagine interne (ovvero per gli utenti pi\`u accomodanti), si ha il 42\%. Per chi visita l'homepage pi\`u spesso (ovvero non \`e la prima volta che accede all'home - un utente abituale) la percentuale di scroll \`e pari al 14\%. Gli utenti detestano in modo esponenziale il numero di scroll necessari\footnote{Pi\`u scroll da fare son presenti, moltissimi pi\`u utenti vengono persi durante la navigazione}. Lo scrolling \`e legato alla grandezza dello schermo in uso: il trend di grandezza degli schermi \`e passato inizialmente da 1024x768 a una riduzione graduale dello schermo, a causa della nascita inizialmente dei Netbooks (1024 x 600) e infine dei telefoni. Va considerato anche che non tutti gli utenti massimizzano le finestre, anche su schermi grandi, causando quindi l'utilizzo di una taglia media di sicurezza che \`e \textit{800x600}. Un aumento di questa grandezza di risoluzione pu\`o causare problemi. Con la nascita dei telefonini ci si \`e accorti che la risoluzione non conta pi\`u: infatti i nuovi telefoni hanno un'alta risoluzione, nonostante lo schermo sia comunque piccolo: ci\`o causa un peggioramento della leggibilit\`a (con lo schermo pi\`u piccolo si hanno caratteri pi\`u piccoli). Inoltre, quando si disegna con una dimensione di schermo fissa, bisogna stare attenti a un altro problema: il \textit{frozen layout}. Il frozen layout \`e il duale del problema dello scrolling, e causa uno spazio vuoto nella pagina web. Il problema dello scrolling in generale \`e che \`e presente troppa informazione su un certo asse visuale: nel monitor son presenti due assi, quello orizzontale e verticale. Quando l'informazione non interagisce con l'asse $x$ si hanno problemi di frozen layout. In questo modo, quando la larghezza della finestra esce da questi parametri, si hanno seri problemi di visualizzazione, ma non solo: pu\`o accadere l'opposto, ovvero che si sfori sull'asse $y$, problema molto accentuato con i telefonini che hanno uno schermo piccolo.
Nei telefoni, lo scrolling \textit{orizzontale} \`e per\`o peggiore: lo scroll verticale \`e accettabile sui palmari, mentre lo scroll orizzontale non \`e comune e non rispetta il nostro modo di ricevere il materiale informativo\footnote{Questo discorso non vale per tutti: per le culture arabe soffrono del problema opposto}. Un altro problema dello scroll sull'asse $x$ \`e che aumenta gli assi informativi da uno spazio a 1 dimensione si passa a uno a 2 dimensioni, con conseguente aumento dello sforzo computazionale).
