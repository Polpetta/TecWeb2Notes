\section{Lezione del 12-11-15}


\paragraph*{Modalit\`a di presentazione dei risultati}Nel presentare i risultati di ricerca ad un utente solitamente \`e meglio adottare una presentazione lineare. Una modalit\`a alternativa \`e a griglia, che presenta il vantaggio di contenere pi\`u informazione e di fornire una rappresentazione compatta a livello visivo, ma tende a far perdere la rilevanza dei risultati, causando disorientamento all'utente con conseguente perdita di tempo.

\subsubsection{Struttura del box search}\`E importante tenere conto della grandezza della casella di ricerca e del modo di cercare degli utenti: si \`e infatti notato che rispetto alle prime ricerche (che si basavano su keyword) ora si \`e passati a query pi\`u complesse. La lunghezza media consigliata per la grandezza di una casella di ricerca \`e di almeno 30 caratteri, in quanto il 90\% delle query presenta una lunghezza simile. In caso il box di ricerca sia troppo piccolo si hanno diversi problemi:
\begin{itemize}

\item Effetto psicologico: lo stress dell'utente aumenta proporzionalmente in base alla quantit\`a di testo non visualizzato nel box di ricerca (1\% di stress in pi\`u per ogni carattere)

\item Effetto perverso sulla ricerca: su un box piccolo gli utenti tendono a scrivere meno quindi il motore ha una ricerca meno precisa e i risultati ottenuti sono pi\`u scarsi.

\end{itemize}


