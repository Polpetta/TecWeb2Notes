\section{Lezione del 08-10-15}

\subsection{Riassunto lezione precedente}
Importantza del tempo nelle pagine web. Quello che vogliono gli utenti sono i 6 assi informativi del giornalismo (where who why what when how)

Tempo in media: 31 secondi nella home, 53 secondi nelle pagine interne.

Il tempo di scelta: oltre il tempo di pagina c'\`e un timer globale complessivo di due minuti in cui l'utente decide se rimere o andare via nel sito.

L'utente si aspetta di aver trovato quello che cerca in circa 3 minuti e 49 secondi.

\subsection{Timer e tempistiche di navigazione}

Visto il limite di scelta di 1m e 49 secondi significa che l'utente dopo aver visto l'home page e navigato poco pi\`u di una pagina interna, fa la sua scelta, se continuare da noi o cambiare sito.

Con il timer globale di 3 min e 49 secondi ci si aspetta di finire quello che deve fare (l'utente) e in media vengono visitate tre pagine e mezzo del sito.

\subsubsection{L'importanza della struttura}

La struttura di un sito diventa critica, non \`e cos\`i banale. \`E importante considerare la distanza dalla nostra home page.

Quali assi (6 domande) dobbiamo tenere all'interno del sito? Cosa diamo all'interno delle nostre pagine? Potremmo concentrarci su altro, ma se ci concentriamo su altro sbagliamo: infatti la navigazione \`e cambiata, tempo fa la navigazione cominciava sempre dalla home page, ai giorni nostri questo non \`e pi\`u vero. La causa di tutto ci\`o sono i motori di ricerca che funzionano sempre meglio, e che permettono che la navigazione cominci da qualsiasi punto del sito! In gergo tecnico questa cosa si chiama \textbf{deep linking}. Quindi ogni pagina potrebbe essere la prima pagina, e la situazione si fa pi\`u complessa: ogni pagina pu\`o essere la pagina iniziale che un utente vede.

Agli assi informativi succede che parte delle informazioni che abbiamo descritto nella home andrebbero riscritte per ogni pagina del nostro sito. Fortunatamente la situazione non \`e proprio cos\`i: alcune informazioni sono opzionali, mentre altri assi sono obbligatorie. Assi opzionali:
\begin{itemize}

\item L'asse when

\item L'asse why mi permette di capire qual'\`e il focus del sito.

\item L'asse how, che tipicamente non sapendo cosa l'utente altro vuole dal sito il designer deve offrire una search, in caso l'utente volesse cercare qualcos'altro. La posizione migliore \`e in alto a destra

\item Se abbiamo spazio, \`e utile mettere anche dei link alle pagine correlate

\item L'asse Where diventa ancora pi\`u importante perch\`e ora l'utente \`e catapultato in mezzo al nostro ``bosco'' informativo. Tipicamente, si dovrebbe rendere chiaro il contesto (la ``minimappa'' o mappa locale) dove si trova (il ``territorio circostante''). Si potrebbe obiettare che \`e gi\`a presente l'asse what, riprendere il link che porta alla home e navigare da li tramite gli altri link, ma questo implica fare passaggi in pi\`u all'utente, e questo fa perdere un click all'utente. Occorrerebbe dare questa informazione sul web direttamente alla pagina. Se arriva ad una certa pagina, avete gi\`a molta informazione su cosa cerca, inutile perderla rispedendolo alla home.

\end{itemize}

Assi invece obbligatori:

\begin{itemize}

\item Who (dove sono? \`E importante indicare il logo)

\item What (tipicamente, link diretto alla home page. Dev'esserci un modo semplice per tornare alla home e capire che sito \`e)

\end{itemize}

In generale, l'asse Where in termine tecnico viene detto \textbf{breadcrumbs}\footnote{Piccoli pezzi di pane}. Ci sono 3 tipi primari di breadcrumbs:
\begin{itemize}

\item Location. Ci da il posto della pagina nella gerarchia del sito. \`E una cosa molto classica, e solitamente questa minimappa \`e anche navigabile.

\item Attribute. Invece di diviere la pagina in un grande albero delle categorie mostrano gli attributi della pagina. Qui si mostrano gli attributi che possono non corrispondere a una categoria di attributi. Pu\`o essere che una pagina sia presente in pi\`u categorie, diversamente dalla Location in cui una pagina ha una sola categoria. Risulta molto pi\`u flessibile. Sembra essere la scelta migliore tra gli altri cammini. Questo per\`o \`e pi\`u costoso per chi fa il sito, ed inoltre la taglia del cammino pu\`o diventare molto grande.

\item Path. Questo tipo di breadcrumbs \`e dinamico in quanto mostra il cammino effettuato dell'utente, richiedendo un carico un po' pi\`u alto nel server. Questo cammino \`e interessante ma non risolve il problema iniziale di essere dispersi in un sito

\end{itemize}

Tutti i breadcrumbs hanno solitamente dei separatori, che sono di solito il maggiore ( \textgreater ) e lo slash ( / ), e sono quelli che gli utenti si aspettano.

\subsection{Problemi di usabilit\`a}

Due tipi di categorie principali: persistenti e non persistenti.

\subsubsection{Persistenti}

Quando si naviga su un sito, occorre sempre stare attenti al problema del \textit{lost in navigation}. Gli utenti devono essere coscienti di dove sono nel sito, cio\`e l'asse Where. Quindi, tipicamente gli utenti devono essere coscienti di dove sono e dove potranno andare, ma oltre ai breadcrumbs \`e necessario dare qualcosa in pi\`u. \`E importante anche che l'utente si ricordi dove ha navigato. In poche parole, occorre fare del nostro meglio per non affaticare l'utente; una semplicissima possibilit\`a \`e quella di colorare i link gi\`a visitati dall'utente\footnote{Un errore dei designer \`e quella di rimuovere il cambio di colore, causando problemi all'utente che naviga nel nostro sito, perch\`e deve compiere uno sforzo computazionale maggiore, anche se peggiora l'esteticit\`a}, infatti il 74\% dei siti web rispetta questa convenzione.

Durante la navigazione su un sito, i movimenti di navigazione sono non solo inizialmente quello dei click, ma anche quelli della pressione del pulsante ``\textit{back}''. Questo secondo movimento non \`e da sottovalutare, perch\`e agli utente piace navigare all'indietro, anche molte volte\footnote{\`E possibile arrivare anche fino a 7 volte}, nonostante ci sia un link diretto. Tutto ci\`o porta a una perdita di tempo (che all'utente interessa), ma non \`e la sua pulsione primaria: lo scopo primario infatti \`e minimizzare la fatica ed evitare lo sforzo. Vantaggi del back button:
\begin{itemize}

\item Non serve ricordarsi il persorso seguito

\item L'interfaccia \`e consistente per tutti i siti (sta fuori dal sito)

\item Non devono cercare dei link

\item \`E consistente con il metodo di cercare ``trial and error''

\end{itemize}

Un \underline{grave} errore quindi \`e quello di \textit{non permettere l'uso del back button}.

Altri modi in cui la navigazione dell'utente pu\`o essere disturbata sono \textit{aprirgli nuove finestre del browser}. Molti designer aprono nuove finestre per separare nettamente nuovo contenuto (o per cercare di tenere l'utente nel proprio sito). Aprire una nuova finestra crea problemi all'utente medio in quanto lo confone e non gli rende pi\`u disponibile l'uso del back button. Le nuove finestre possono essere di due tipi:
\begin{itemize}

\item Finestra a schermo intero: Il problema della nuova finestra, che la rende peggiore del tab, \`e che si sovrappone alla navigazione esistente in maniera non-standard. Se la finestra va a tutto schermo, l'utente medio \`e irritato e confuso: non sa come tornare indietro.

\item Finestra non a schermo intero:\ l'utente medio non chiude una nuova finestra per tornare indietro, ma tipicamente \textit{clicca sulla finestra retrostante}. L'effetto ottenuto \`e che lo schermo si riempie di finestre non volute, e cosa ancora peggiore, se l'utente ritorna in condizioni simili, il link che apre la stessa nuova finestra sembrer\`a non funzionare, quindi occorre fare molta attenzione all'apertura di nuove finestre o tab, anche se in alcune occasioni hanno ragion d'essere. Un problema correlato \`e il \textbf{pop-up}, che non sono altro che nuove finestre, ma aperte \textit{senza il permesso dell'utente}.

\end{itemize}
