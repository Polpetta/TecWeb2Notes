\section{Lezione del 13-11-15}

\subsubsection{Assistenza multimediale}

La assistenza multimediale viene anche detta ``Ricerca 2.0''. Crea un danno al sito del -42\%. L'utente di fronte a un Human Digital Assistant diventa pi\`u esigente e si aspetta un'interazione maggiore e per migliorarne l'esperienza \`e meglio scegliere un avatar che non ricordi una forma umana.
Con l'assistenza multimediale il rumore di fondo \`e la chiave per creare negli utenti una sensazione positiva, rendendo pi\`u caldo e pi\`u gradevole un sistema a lato utente.

\subsection{Visibilit\`a del Sito}

\`E essenziale essere nella top ten dei motori di ricerca per poter essere ``trovati'' dall'utente.

\subsubsection{SERP}

Il posizionamento di un sito web all'interno di un risultato di ricerca \`e molto importante. I primi 10 risultati di ricerca assorbono il 95\% dei click.
Percentuale dei click in base alla posizione:
\begin{itemize}

\item Prima posizione: 51\%
\item Seconda posizione: 16\%
\item Terza posizione: 6\%
\item Quarta posizione: 6\%
\item Quinta posizione: 5\%
\item Sesta posizione: 4\%
\item Settima posizione: 2\%
\item Fino al decimo posto: 2\%. Questo a causa dell'effetto Jearsy (o effetto Malabrocca)

\end{itemize}

