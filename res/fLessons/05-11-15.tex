\section{Lezione del 05-11-15}

\subsection{Pubblicit\`a}

Oggi la pubblicit\`a \`e normale, tempo fa no. Fa parte del \textit{modello di business standard}. Questo viene applicato in un servizio gratis principalmente, in quanto la pubblicit\`a permette di sostentare il sito.

\paragraph*{Reazione degli utenti}Agli utenti non piace la pubblicit\`a, in quanto si frappone ai loro obiettivi di navigazione. Una misura sul click della pubblicit\`a web \`e dello $0,4\%$. \`E importante avere un buon posizionamento all'interno del sito web come nei motori di ricerca, in modo tale che il messaggio pubblicitario sia il quanto pi\`u efficace possibile.

\subsubsection{Posizionamento}Un buon posizionamento, in prima analisi e considerando le coordinate assolute sulla pagina \textit{la colonna di sinistra} \`e un buon punto di posizionamento, seguito dalla striscia\textit{ top della pagina}, dalla \textit{colonna di destra} e infine con la pubblicit\`a \textit{in fondo la pagina} (viene visualizzato 10 volte in meno).
La pubblicit\`a messa \underline{vicino} a contenuto interessante viene visualizzata maggiormente.
La taglia della pubblicit\`a influisce relativamente sulla visibilit\`a del messaggio.

\subsubsection{Metodi di promozione del messaggio pubblicitario}\`E da evitare assolutamente (ordinato in modo crescente) che il messaggio pubblicitario\footnote{Percentuale di non-gradimento}:
\begin{enumerate}

\item Suoni automaticamente ($79\%$)
\item Sia in movimento ($79\%$)
\item Lampeggi ($87\%$)
\item Occupi la maggior parte della pagina ($90\%$)
\item Si sposti sullo schermo ($79\%$)
\item Non dica di cosa si tratti ($92\%$) $\to$ gambling click
\item Copra quello che si sta cercando di leggere ($93\%$)
\item Non abbia un modo chiaro per toglierlo ($93\%$)
\item Cerchi di farsi cliccare sopra ($94\%$)
\item Si carichi lentamente ($94\%$)
\item Sia un pop-up ($95\%$)

\end{enumerate}
Quindi devono essere usati altri ``effetti speciali''.
Le immagini pubblicitarie rispondono alle dinamiche delle immagini nel web, rimanendo in serie B. Questo a causa dell'effetto \textit{``zapping''}, in cui viene applicata da parte del cervello umano una meccanica automatica in cui vengono ``eliminate le scocciature''. Questo effetto viene accentuato se l'immagine \`e una pubblicit\`a.
La regola dello zapping si applica anche al testo quando esso \`e troppo grande o troppo visibile, e per superare la barriera dello zapping bisogna andare controcorrente, tramite tecniche studiate a tavolino. In questa maniera vengono saltati gli algoritmi pre-impostati nel cervello attirando l'attenzione degli utenti, riportando gli avvisi pubblicitari in ``serie A''. Quindi \`e necessario \textit{confondere le idee} agli utenti\footnote{Ovvero scardinare gli algoritmi di zapping.}, possibilmente mescolando testo con il contenuto pubblicitario.
I \textit{giochetti web} catturano l'attenzione degli utenti e sono iper-efficaci per la pubblicit\`a, azzerando i timer. \`E importante quindi avere giochi semplici che aiutino anche l'utente con infine l'inserzione pubblicitaria.
Il modo migliore per creare pubblicit\`a \`e il testo, che \`e il miglior \textbf{blending}. Altri effetti se si vogliono usare immagini in maniera efficace \`e creare immagini pubblicitarie in cui le persone (nell'immagine pubblicitaria) guardino l'oggetto d'interesse o di mostrare persone a grandezza naturale, in quanto prima viene analizzato il viso e poi gli organi genitali\footnote{Questo viene in maniera inconscia.}, ed \`e quindi utile mettere la pubblicit\`a in quel punto.

\subsubsection{Contenuto e contestualizzazione}Il contenuto \`e importante, in quanto se mostro oggetti scollegati dal sito l'utente ha un senso di disorientamento/distrazione. Nel sito web questo effetto \`e ancora maggiore, in quanto ogni sito \`e spesso equivalente a una visita specializzata, con una diminuzione del timer dell'utente fino al $-40\%$, e un calo della voglia di ritornare dell'utente del $-80\%$. \`E importante quindi che il contenuto del messaggio pubblicitario sia ``in armonia'' con il sito web anche detto \textit{behaviorial advertising}, cio\`e la ``pubblicit\`a comportamentale'', che cerca di dare un contenuto che interessi all'utente. A differenza di media passivi, nel web \`e possibile diversificare gli utenti, e quindi dare loro pubblicit\`a ``behavioral'' adatta al loro comportamento. L'efficacia di fare la pubblicit\`a behavioral ha un incremento del fattore di interesse di almeno $10x$, fino a $10000x$ volte! La pubblicit\`a behavioral ha anche un effetto benefico sul sito: non solo non subisce gli effetti negativi del \textit{``distracting advertisement''}, ma addirittura in certi casi li ribalta, portando ad un aumento dei timer e della voglia di ritorno.

