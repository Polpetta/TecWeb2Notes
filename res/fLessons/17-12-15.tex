\section{Lezione del 17-12-15}

Dopo gli assiomi di tipo 0, esistono gli assiomi di tipo 1:
\begin{itemize}

\item[Assioma 1]Non importa a chi e dove specifichi quest'URI, dovr\`a avere sempre lo stesso significato
  
\end{itemize}

\paragraph*{OWL}Il supporto di base fornito da RDF-Schema \`e stato quindi esteso, con uno strato apposito nella ``torre semantica'' dedicato apposta alle ontologie, ovvero \textit{OWL}\footnote{Acronimo per Web Ontology Language.}.
In OWL sono presenti le seguenti funzionalit\`a di uguaglianza:
\begin{itemize}
  
\item equivalentClass
\item equivalentProperty
\item sameIndividualAs (sameAs)
\item differentFrom
\item allDifferent
  
\end{itemize}

OWL permette di ridurre la piaga potenziale dell'URI variant law, stabilendo relazioni tra ontologie diverse. \`E possibile anche stabilire delle propriet\`a per le caratteristiche, come:
\begin{itemize}
\item inverseOf
\item TransitiveProperty
\item SymmetricProperty
\item FunctionalProperty $\to$ per esempio corrispondenza (1,1) tra madre e figlio
\item InverseFunctionalProperty
\end{itemize}
OWL permette di impostare ulteriori restrizioni sui tipi delle propriet\`a, come:
\begin{itemize}
\item allValuesFrom
\item someValuesFrom
\item minCardinality $\to$ si pu\`o impostare la cardinalit\`a minima delle classi.
\item maxCardinality
\item Cardinality
\end{itemize}

Dopo la definizione delle relazioni (ovvero attraverso la logica di funzionamento) bisogna occuparsi dell'implementazione. Questo perch\`e gi\`a la logica del primo ordine (per ogni, esiste un) non \`e decidibile. SQL-92 \`e un esempio di primo linguaggio relazionale, ma non Turing-completo. Nel web semantico si \`e tentato SPARQL\footnote{SPARQL Protocol and RDF Query Language}, che segue alcune scelte di design di SQL.
L'idea di base di SPARQL \`e tramite la conoscenza a triplette (predicato-soggetto-complemento), ricalcando la stessa struttura della query SQL:
\begin{verbatim}
PREFIX ...
SELECT ...
FROM ...
WHERE{
  ...
}
ORDER BY ...
\end{verbatim}

Il PREFIX serve semplicemente a creare prefissi leggibili per i nomispazio, come ad esempio:
\begin{verbatim}
PREFIX foaf:<http://xmlns.com/foaf/0.1/>
\end{verbatim}

\paragraph*{Modellazione con SPARQL}

Esistono due modelli principali, \textit{DC} e \textit{FOAF}.

\begin{itemize}
\item[Dublin Core] \`E praticamente lo standard per descrivere le propriet\`a di base dei documenti, e presenta 15 elementi informativi:
  \begin{itemize}
  \item Creator
  \item Subject
  \item Description
  \item Publisher
  \item Contributor
  \item Date
  \item Type
  \item Format
  \item Identifier
  \item Source
  \item Language
  \item Relation
  \item Coverage
  \item Rights
  \end{itemize}
  Questo insieme compatto permette di definire in modo completo le pagine web. Per esempio in HTML:
\begin{verbatim}
  <meta NAME="DC.title" content="A cosa serve OWL" >
\end{verbatim}

\item[Friend Of A Friend]\`E l'ontologia per il web sociale. Definisce le caratteristiche principali riferendosi a persone con una classe \textit{Person} con diverse propriet\`a. Una delle possibili propriet\`a \`e anche \textit{myersBriggs}, dove sono presenti 16 tipi di personalit\`a distribuite su 4 assi (Estroversione, sensitivit\`a, ragionamento, giudizio). Un esempio:
\begin{verbatim}
<foaf:Person>
  <foaf:name>Massimo Marchiori</foaf:name>
  <foaf:mbox rdf:resource=mailto:massimo@w3.org />
</foaf:Person>
\end{verbatim}

\`E possibile, usando foaf, esprimere anche il concetto di amicizia con \textit{foaf:knows}
\end{itemize}

%PlanetRDF \`e un sito ``semantico'' costruito con il web semantico che sfrutta SPARQL. <- esempio inutile

SPARQL permette anche la gestione dei dati opzionali, importante in un'ottica di Web Semantico, dove i grafi della conoscenza sono costruiti nel/dal Web, quindi possibilmente \textit{parziali}. I dati parziali vengono gestiti tramite la keyword \textit{OPTIONAL}, che indica l'esistenza o meno di dati. Ad esempio:
\begin{verbatim}
 PREFIX foaf:<http://xmlns.com/foaf/0.1/>
 SELECT ?name ?picture
 WHERE{
   ?someone foaf:name ?name .
   OPTIONAL{
    ?someone foaf:picture ?picture .
   }
 }
\end{verbatim}


\paragraph*{Complessit\`a ed efficenza}L'efficenza di SPARQL \`e legata alla complessit\`a. Nella gerarchia della complessit\`a, SPARQL si trova in PSPACE\footnote{Tra i problemi NP e EXPTIME.} (analogamente a SQL).

Con OWL si \`e deciso di dare creare versioni leggermente diverse in base alla complessit\'a desiderata. In ordine di potenza crescente:
\begin{itemize}

\item OWL Lite $\to$ il pi\`u leggero e il pi\`u efficente. \`E \textit{decidibile} e corrisponde alla logica detta SHIF.
\item OWL DL $\to$ \`E \textit{decidibile}, e corrisponde alla logica detta SHOIN (logica descrittiva, DL = Description Logic)
\item OWL Full $\to$ non ha restrizione, ed \`e \textit{indecidibile}
  
\end{itemize}

La versione di OWL Lite appartiene alla complessit\`a logica di EXPTIME, e la SHOIN in NEXPTIME. Queste scelte sono state effettuate in base alle classe di complessit\`a, rimandendo comunque misure statistiche\footnote{``Le statistiche sono come i bikini. Ci\`o che rivelano \`e suggestivo, ma ci\`o che nascondono \`e pi\`u importante'' [cit. Aaron Levenstein]}. In realt\`a la classe di complessit\`a di un problema \`e la punta dell'iceberg e indica la complessit\`a media.

%query exptime - da cercare e provare su google

SPARQL \`e in PSPACE, ma gi\`a solo togliendo la keyword OPTIONAL, scendiamo al 100\% dentro co-NP. Statisticamente, molta della conoscenza che si trova nel web sta in sotto-logiche di OWL, ad esempio AL e ALC, che hanno rispettivamente complessit\`a PSPACE e P.

\paragraph*{Linked Data}Linked Data \`e il nuovo nome per il Semantic Web e dato il suo successo viene usato anche come generalizzazione per indicare dati generalmente utilizzabili per il grafo della conoscenza.

\paragraph*{LOD}I LOD sono un tipo particolare di Linked Data: i Linked Open Data, dove la loro caratteristica principale \`e di essere gratis. Segue una classificazione LOD [il numero indica la stellina]:
\begin{enumerate}

\item Dati disponibili sul web ma con una licenza open (open data)
\item Dati disponibili sul web in un formato strutturato machine-readable
\item Come prima, usando un formato dati non proprietario
\item Come prima, ma in formato Semantic Web (RDF*, OWL)
\item Come prima, con dati linkati ai dati di altri per dare il contesto
\end{enumerate}

%star badges -> da guardare

La fascia con tre stelle (dati in formati non proprietari) non sono necessariamente in RDF/OWL, ma \`e possibile trasformarli tramite l'uso di macchine. Quando si passa da un mondo non Semantic Web (da tre stelle in gi\`u) al mondo RDF si parla di \textit{lifting}, mentre il passaggio inverso si dice \textit{lowering}.
Lifting e lowering sono estremamente importanti per fare il cosiddetto \textit{mashup}, cio\`e unire il mondo XML/HTML/XHTML con RDF/OWL.
