\section{Lezione del 23-10-15}

Non bisogna mettere troppo alla prova la fiducia dell'utente e bisogna tenere a mente sempre l'asse who, che attiva il trust su di un sito. Tutto ci\`o crolla rapidamente quando si usano varie tecniche in uso nella pubblicit\`a comune: nella pubblicit\`a classica l'utente \`e a contatto con il messaggio pubblicitario per poco tempo, impressionandolo e colpendolo. Si pu\`o compiere in vari modi e uno dei modi \`e giocare sul prezzo molto favorevole. Ci sono almeno due trucchi classici: \textit{fishing price} dove c'\`e un ``prezzo esca'' che non corrisponde praticamente mai il prezzo reale e \textit{net price} dove si indica un prezzo netto che non corrisponde al tutto incluso (al prezzo complessivo finale). In entrambi i casi il prezzo comunicato \`e uno, mentre quello offerto \`e un altro.

Il fatto che Internet sia un posto ``virtuale'' porta molti a confondere purtroppo questi due concetti, ed a pensare che gli utenti si possano gestire con i classici trucchi pubblicitari. Il fatto che questi approcci non funzionino in determinate casistiche dipende dal funzionamento del cervello: la memoria si divide in gestione a breve e a lungo termine e il cervello applica dei filtri per determinare il tipo di memoria su cui deve andare il ricordo. La pubblicit\`a nel mondo normale sfrutta la short memory e porta solo ad attrarre clienti nel negozio. Questa tecnica non funziona in un sito web e quando viene applicata causa irritazione all'utente.
Gli effetti del fishing price in un sito web provocano un abbandono del $90\%$ degli utenti, mentre il $10\%$ degli utenti che rimangono subiscono un grave calo nel trust, con frustazione e diminuzione dei timer del $50\%$.
Gli effetti del net price in un sito web (il trucco classico \`e di non indicare l'iva) provoca un abbandono dell'$85\%$ degli utenti. Fare net price \`e possibile farlo anche in modo involontario: per esempio sulle spese di spedizione di un prodotto. Una soluzione potrebbe essere di dare l'effettivo prezzo al \textit{checkout}, ma questo causa gambling clicks da parte dell'utente perch\`e non lo pu\`o sapere fino alla fine. \`E importante mettere le spese di spedizione insieme al prezzo, perch\`e \`e questo quello che conta all'utente.

Se offriamo qualcosa di gratuito (ad esempio come le spese di spedizione) \`e molto importante sottolinearlo bene nel sito con la parola chiave ``gratis'', in quanto \`e un attivatore di sensazioni positive.

\subsubsection{Presentazione dei prodotti}

\paragraph*{Descrizione visiva}Come con il prezzo, \`e importate chiarire subito di che prodotto si tratta, evitando gambling clicks. Un errore tipico e grave \`e quello di assumere che l'utente conosca gi\`a il prodotto e venga sul sito solo a conoscerne il prezzo. Questo \`e un atteggiamento sbagliato perch\`e l'utente si aspetta una \textit{descrizione completa}, con quante pi\`u informazioni possibili e questo pu\`o portare l'utente a navigare in altri siti aldil\`a del prezzo che si offre; anche se si \`e il miglior venditore se non \`e presente una buona descrizione si ha una impressione negativa da parte degli utenti. Cattive descrizioni portano il $99\%$ degli utenti ad abbandonare il sito e migrare su altri anche se la nostra offerta \`e fino al $20\%$ pi\`u vantaggiosa (solo il $5\%$ ritorna).

\paragraph*{Aspetto visivo}Dare una descrizione visiva \`e importantissima. In questi casi inoltre i timer praticamente si spengono: quando \textit{sceglie}\footnote{Attenzione quindi a non forzare una visione troppo amplia in partenza, con costi sul timer.} di avere pi\`u dettagli, l'utente aspetta volentieri per una immagine pi\`u dettagliata. L'ideale \`e quindi offrire varie prospettive di vista ad alto dettaglio, ricalcando la linea guida sul \textit{3d come 2d}\footnote{Argomento discusso nelle lezioni precedenti.}. Il dettaglio delle immagini dipende anche dal contesto: oggetti particolari potrebbero richiedere foto con zoom maggiori o particolare attenzione ai dettagli.

\subsection{Analisi del comportamento utente}

\`E stata fatta l'analisi del comportamento degli utenti rispetto a vari media, facendo emergere classi di comportamento ben preciso, confermando l'esistenza di \textbf{regole generali}. Diversi comportamenti in base al media:
\begin{itemize}

\item \textit{Giornale}: c'\`e un punto di attrazione principale che coinvolge subito: le foto e successivamente il colore perch\`e viene percepita l'immagine come pi\`u ricca di informazione. Nel giornale devono essere predominanti le immagini, in quanto sono pi\`u viste del testo in una proporzione $80\%$ a $20\%$. Pezzi di testo accompagnati da un'immagine sono visti molto di pi\`u che quelli senza un'immagine che li accompagna. Il punto di entrata nella pagina di un giornale \`e dall'immagine pi\`u grande. Due pagine aperte sono percepite come un'unica grande pagina in formato panoramico.

\item \textit{Web}: quello che si \`e pensato per molto tempo \`e che il Web seguisse le stesse regole del giornale, ma il Web \`e un media completamente diverso, ed in questo media gli utenti si comportano diversamente. I punti di entrata di attrazione in una pagina web sono circa a partire dalla parte in alto a sinistra dello schermo. Tipicamente la forma classica \`e detta \textbf{a F} o \textbf{a cono gelato}. Gli utenti in stragrande maggioranza effettuano uno scroll schermata per schermata, seguendo diverse impronte termografiche sulla prima schermata e sulla seconda. Questo effetto crea ``zone cieche'' che ricevono poca attenzione dagli utenti e questo dipende dalla taglia dello schermo, ed \`e un problema molto serio.
Nel caso del web tra testo e immagini si ha una vittoria del testo, e le immagini vengono poste in ``serie B''.


\end{itemize}
