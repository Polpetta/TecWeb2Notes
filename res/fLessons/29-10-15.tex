\section{Lezione del 29-10-15}

\paragraph*{Visualizzazione e impaginazione pagine web}Il \textit{punto di attrazione} nelle pagine web si trova in alto a sinistra. Pi\`u specificatamente \`e il testo che sta nell'angolo, quindi paradossalmente mettere un logo senza testo fa disorientare gli utenti, in quanto i loro occhi si aspettano del testo da leggere.
L'impaginazione di un sito web deve avvenire su una colonna\footnote{I media classici solitamente impaginano a pi\`u colonne per esempio i giornali} perch\`e il numero di colonne affatica l'utente in quanto si utilizza l'asse orizzontale.

\paragraph*{Parole chiave}Le parole chiave che stanno sulla stessa linea tendono ad non essere facilmente percepite durante lo scan, ed usare il bold per le parole chiave risulta essere poco efficace. Possibili soluzioni sono porle su un'unica linea (come i titoli), renderle pi\`u grandi oppure sottolinearle come per gli hyperlinks.

\paragraph*{Titolo e testo}Il pezzo di testo pi\`u attrattivo per gli utenti \`e collegato alla lunghezza: nella ``lotta'' tra paragrafi vince il pi\`u corto, che arriva ad attrarre almeno il doppio di uno lungo. Similmente anche i titoli seguono questa logica: meglio l'utilizzo di titoli corti rispetto a titoli lunghi. Avere un testo spezzato con molti paragrafi corti rilassa i timer ed incoraggia alla lettura (i timer vengono rilassati fino ad un 100\%).

Per migliorare un titlo si aggiunge un \textbf{blurb}, ovvero una spiegazione sotto il titolo, un sommario di secondo livello cio\`e un piccolo paragrafo che spiega il contenuto del testo. Il blurb aumenta la mappa termografica e rilassa i timer, ma questo non cambia molto la voglia di un utente a proseguire la navigazione; in ogni caso fa aumentare il tasso di ritorno degli utenti alla stessa pagina (fino a un +20\%).

Il posizionamento del testo deve seguire regole precise:
\begin{itemize}

\item Il contenuto del blurb non \`e un blocco unico per gli utenti: pu\`o essere diviso in varie zone come una pagina web. Essendo percepito assimetricamente la parte sinistra di un blurb risulta essere pi\`u importante ed \`e quindi importante che le giuste parole stiano sul lato sinistro
\item Occorre trovare il giusto compromesso tra seprarazione e compattezza dell'informazione: eccessiva seprarazione non implica pi\`u usbilit\`a, ma provoca il \textit{diluited design}. Esistono delle regole per ottenere un giusto compromesso:
  \begin{itemize}

  \item La separazione porta a scanning pi\`u veloce, utile per pagina di navigazione con link e poco scroll
  \item Per pagine con contenuto e scroll, la separazione funziona male e crea diluited desgn

  \end{itemize}

\end{itemize}

\paragraph*{Immagini}Le immagini risultano di serie B rispetto al testo, ma sono comunque importanti: devono seguire una grandezza minima di 210x230 pixels per il desktop, altrimenti vengono percepite dagli utenti come delle icone e non trasmettono l'informazione.
Le immagini dispongono di una propriet\`a che le rende interessanti: creano un effetto calamita per i click: si \`e visto che il 20\% degli utenti clicca sulle immagini, rendendo opportuno che le immagini siano dei link a punti correlati del sito. Un esempio particolare sono le slideshow: un utente si aspetta sempre che cliccando le immagini succeda qualcosa.

\subsection{Legge di Fitts}

\`E stata studiata la velocit\`a di spostamento dell'utente in una pagina web, che risulta essere:

\[ T=a+b \cdot \log_2 {(1+ \frac{D}{W})} \]

Dove:
\begin{itemize}

\item $a$: start/stop $\to$ il tempo che l'utente impiega per iniziare/terminare un'azione
\item $b$: covelocit\`a $\to$ inverso della velocit\`a che l'utente ha per spostarsi con il mouse
\item $D$: distanza $\to$ distanza che l'utente deve percorrere
\item $W$: ampiezza $\to$ grandezza del target da cliccare

\end{itemize}

\`E importante notare che anche se ho elementi distanti non \`e  cos\`i importante, ma se l'oggetto \`e piccolo, questo influisce negativamente per i tempi.

\textit{Nota bene}: il drag and drop ha una covelocit\`a pi\`u grande perch\`e c'\`e tensione muscolare, quindi andrebbe evitato nelle interfacce web.
