\section{Lezione del 22-10-15}

Ci sono delle precauzioni che dovremmo considerare quando inseriamo del blocco di testo, come ad esempio evitare il caps lock che nella netiquette corrisponde all'``urlo''. Il testo tutto maiuscolo si legge pi\`u lentamente e ci vuole minimo il 10\% in pi\`u per leggere un blocco di testo in caps lock rispetto a uno normale. Questo perch\`e di solito la mente umana processa testo scritto in minuscolo, e questo causa un rallentamento nella lettura e un aumento di fatica computazionale, che lascia una sgradevole sensazione.

Un'altra tentazione \`e quella di usare grafica al posto del testo e questa soluzione porta a vari problemi:
\begin{itemize}

\item Il testo non scala (o se scala sgrana)
\item Aumenta il carico della pagina (diventa pi\`u pesante)
\item Non permette il copia-incolla
\item Interagisce male con i motori di ricerca, in quanto per i motori di ricerca attuali una immagine \`e una ``scatola nera'' che non interagisce con l'immagine.

\end{itemize}

\paragraph*{Maledizione del Lorem Ipsum} La tecnica del Lorem Ipsum\footnote{Deriva da un testo di Cicerone ``I limiti del bene e del male''.} \`e una tecnica fallimentare insegnata in corsi di Web Design, dove viene disegnato il layout e il testo viene riempito temporaneamente con il testo del Lorem Ipsum. Il vantaggio di questa tecnica \`e la modularizzazione del layout visivo con il contenuto, ma in questo modo il layout visivo diventa la priorit\`a e il testo diventa una componente secondaria, mentre dovrebbe essere il contrario. \textit{Per strutturare il testo bisogna avere contenuto}. Un altro problema generato dall'uso di questo design \`e il \textit{testo ghigliottinato} causato da un troncamento improvviso.

\paragraph*{Azione degli utenti all'interno di una pagina web} All'inizio della navigazione, l'utente entra in una modalit\`a di ``scanning'' della pagina in cui viene data una visione veloce del sito. Ci\`o comporta una visita non ordinata della pagina dove vengono analizzate le componenti in maniera veloce, e viene creata una immagine mentale dell'informazione. Questo scan avviene in maniera continua utilizzando gli occhi e ci\`o comporta che un buon design minimizzer\`a lo sforzo di scanning degli utenti, tenendo conto che viene fatto come \textit{una linea continua}.
Quando l'occhio incontra grossi blocchi testuali non riesce a fare lo scanning veloce e si ha un aumento di fatica perch\`e non si riesce a mapparlo, ma rimane un'incognita. Se invece non c'\`e un unico blocco di testo allora \`e possibile creare le varie associazioni mentali. Occorre quindi dare una struttura al testo per dar modo agli utenti di modellare una buona mappa informativa.

\paragraph{Strutturazione del testo} Sono presenti vari modi per strutturare il testo, ed \`e meglio strutturare testo in blocchi fini. Un modo per strutturare meglio il testo \`e:
\begin{itemize}

\item Tramite \textbf{titoli descrittivi}: in generale, ogni blocco di testo rilevante dovrebbe avere un titolo significativo, e questo aiuta a creare una categorizzazione primaria
\item Le parole chiave all'interno del testo (\textbf{keyword}) devono essere brevi e pertinenti, rendendo lo scanning facile e veloce: infatti ogni blocco viene ``saltato'' e solo la keyword assimilata\footnote{Un esempio in twitter sono gli hashtag: servono per evidenziare le parole chiave in fase di scan}. Ad esempio i link si notano come e pi\`u delle parole chiave e dovrebbero seguirne gli stessi principi (evitando nomi uguali ed il ``clicca qui'').

\item Attraverso l'utilizzo di \textbf{liste itemizzate}, ordinate o no. Migliorano il grado di soffisfazione dell'utente del $+47\%$! Esse vanno usate tipicamente quando si hanno almeno $4$ elementi. Le liste che hanno meno di tre elementi generano affaticamento all'utente, producendo un effetto contrario. Un utilizzo eccessivo di liste disposte verticalmente genera un decadimento \textit{lineare} della loro efficacia e \textit{esponenziale} con il numero di liste disposte orizzontalmente

\item Evitare lo scroll interno al sito, in quanto \`e devastante rispetto allo scroll esterno del browser. Uno scroll interno indica una progettazione fallimentare nella creazione del sito

\end{itemize}

L'``effetto bionda'' \`e quell'effetto per cui durante la fase di scanning si ha una percezione completamente sbagliata di un oggetto nella pagina che viene considerato in maniera negativa dall'utente.

\subsection{Siti commerciali}

I siti (e-)commerce hanno delle particolarit\`a e la cosa pi\`u importante in un tale sito \`e il prodotto che si vende, ma non \`e l'unica. Per l'utente medio, la componente pi\`u importante, alla pari del prodotto, \`e \textit{il prezzo}. La regola fondamentale da tenere a mente \`e che gli utenti vogliono sapere il prezzo del prodotto in modo semplice. Il posto in cui il prezzo va inserito \`e quanto pi\`u possibile vicino al prodotto, come accade nei negozi fisici.

L'iper-associazione causa molto fastidio agli utenti in quanto per ricevere informazini sul prezzo \`e necessario vedere i dettagli del prodotto, causando un alto costo computazionale. Gli utenti infatti tendono ad associare il prodotto con il prezzo e la iper-associazione provoca mancanza di informazione primaria. L'utente \`e costretto ad eseguire un \textbf{gambling click} per completare l'informazione ``prodotto-prezzo'': questo genera un grande stress mentale ($-40\%$ di gradimento) e non attira gli utenti. Se un utente non \`e fidalizzato, solamente il 30\% clicca su un gambling click.

Uno dei problemi sui prezzi pu\`o essere che alle volte non c'\`e un prezzo definito, causando un fallimento completo perch\`e all'utente interessa principalmente il prezzo. La soluzione \`e dare un prezzo in ogni caso, e quando non \`e disponibile \`e meglio dare un prezzo approssimato o un range di prezzi.
